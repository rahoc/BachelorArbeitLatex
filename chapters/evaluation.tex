\chapter{Evaluation}

Das Ziel der Evaluation ist die Performance des Fissions-Algorithmus zu messen, welcher die Bewertungsfunktionen für verschiedene Dialog-Outputs berechnet. Wie in der Theoretischen Betrachtung festgestellt sind dabei folgende unabhängigen Variablensets zu beachten: \emph{Personmodel}, \emph{Surroundingsmodel}, \emph{DistanceRelations}, die \emph{Device}- und \emph{Componentmodel} Permutationen, \emph{Rules} und der jeweilige \emph{DialogOutput}. Zudem gilt es die Hardwarekonfiguration des Testsystems zu beachten.
Abhängige Variable ist die Zeit, die der Algorithmus zur Berechnung benötigt.

\section{Entwicklung des Tools zur Evaluation}
Um das Evaluationsvorhaben durchzuführen wird ein zusätzliches Tool entwickelt. Es hat insbesondere drei Aufgaben: Erzeugen von generischen Testdaten (unabhängige Variablen), Versand der Daten als Semaine-Nachrichten an die Fission und das Logging der Testergebnisse (abhängige Variablen).

\subsection{Generische Testdaten}
Bei der generischen Erzeugung der Testdaten gelten spezielle Anforderungen. Mit der Klasse \emph{Random}, die durch das .Net Framework zur Verfügung gestellt können Zufallszahlen erzeugt werden. Jedoch bietet diese Klasse keine Funktionen an um Wahrscheinlichkeitsverteilungen zu erzeugen. Für die Erzeugung dieser Zufallswerte \todor{auf Zufallswerte eingehen} wurde die statische Klasse \emph{Randomizer} erstellt. Diese bietet verschiedene Methoden an um die generischen Testdaten zu erzeugen.
\subsubsection{Zufällige, gleichverteilte Wahrscheinlichkeitswerte}
Eine dieser Anforderungen war es, für beliebig viele Entitäten eine Wahrscheinlichkeitsverteilung zu erzeugen, die in Summe eine vorgegebene Gesamtwahrscheinlichkeit ergibt. Die Einzelwerte sollen hierbei jedoch zufällig gleichverteilt und \emph{nicht} normalverteilt sein. Eine mathematische Lösung für dieses Problem bietet folgender Ansatz\footnote{http://www.matheboard.de/archive/501805/thread.html, 14.11.2013}:
\begin{enumerate}
\item Erzeuge \( n \)  verschiedene Zufallszahlen \(z_{i}, i = 1...n \) mit \(z_{i} \in \mathbb{R}^+ \) 
\item Summiere über alle Zufallszahlen: \( z:= \sum_{i=1}^{n} z_{i}\)
\item Teile die gewünschte Summe der Zufallszahlen \( s \)  durch die Summe der \\Zufallszahlen \( z \): \( \frac{s}{z} = m\)
\item Multipliziere alle Zufallszahlen \( z_{1} ... z_{n} \) mit dem Faktor \( m \) und erhalte die gewünschten Zufallszahlen \( y_{1} ... y_{n} \) mit \( s = \sum_{i=1}^{n} y_{i}\) mit \(y_{i} = m * z_{i} \)
\end{enumerate}
Diese Schritte sind im Algorithmus (Listing \ref{RandomDistribution}) umgesetzt. Der Methode müssen ein \emph{Random}-Objekt, die gewünschte Summe der Zufallszahlen (bei Wahrscheinlichkeitsverteilung z.\,B.\ 1 (=100\%)) und die Anzahl der Werte übergeben werden. Das \emph{Random}-Objekt wird an die Methode übergeben, da bei mehrfachem Aufruf der Methode dasselbe Objekt verwendet werden sollte, um eine breite Streuung der Zufallszahlen zu erhalten.

\lstset{caption={Gleichverteilte Wahrscheinlichkeitsverteilung für beliebig viele Werte},label=RandomDistribution}
\begin{lstlisting}
public static List<Double> RandomDistribution(Random rnd, Double sum, Int32 count)
{
    // Liste für die erstellten Werte (Rückgabewert)
    List<Double> randomValues = new List<Double>();
    // Summe der Zufallszahlen
    Double sumZ = 0;

    for (Int32 i = 0; i < count; i++)
    {
        Double randomValue = rnd.NextDouble();
        // Erzeuge die gewünschte Anzahl Zufallswärte
        randomValues.Add(randomValue);
        // Berechne Summe z der erzeugten Zufallswerte
        sumZ += randomValue;
    }

    // Sei sum = x. Teile x/z = m.
    if (sumZ <= 0)
    {
        // Nicht durch 0 teilen, Wert sollte positiv sein!
        return null;
    }
    Double multiply = sum / sumZ;

    // Multipliziere alle Zufallszahlen mit m.
    for (Int32 i = 0; i < count; i++)
    {
        randomValues[i] = randomValues[i] * multiply;
    }

    return randomValues;
}
\end{lstlisting}


\subsubsection{Zufällige Intgerwerte mit gleichverteilten Wahrscheinlichkeitswerten}
Eine weitere Anforderung ist die zufällige Generierung von beliebig, jedoch nur einmal auftretenden Integer-Werten innerhalb eines bestimmten Intervalls. Diese wiederum müssen mit einer Wahrscheinlichkeitsverteilung versehen sein. So soll z.\,B.\ auf einer Skala von eins bis zehn, n zufällige Werte gewählt und mit einer Wahrscheinlichkeit belegt werden. Die Summe der n Wahrscheinlichkeitswerte soll eins ergeben (etwa: 3=0,2; 4=0,1; 6=0,4; 9=0,3). Der in Listing \ref{IntAndProbabilityDistribuion} dargestellte Algorithmus erzeugt eine zufällige Anzahl beliebiger jedoch distinkter Integer-Werte in einem vorgegebenen Zahlenbereich [min, max]. Diesen Zahlen wiederum wird dann mit der im vorangegangenen Abschnitt beschriebenen Methode eine Wahrscheinlichkeitsverteilung zugeordnet. Diese Zuordnung wird in einem \emph{Dictionary<int,double>} gespeichert und zurückgegeben.
\lstset{caption={Gleichverteilte Wahrscheinlichkeitsverteilung für beliebige Integerwerte},label=IntAndProbabilityDistribuion}
\begin{lstlisting}
public static Dictionary<int, double> IntAndProbabilityDistribuion( Random rnd,  int min, int max, double sum) {

	// Rückgabewert
    Dictionary<int, double> distribution = new Dictionary<int, double>();

    List<int> values = new List<int>();
    // Erzeuge eine bestimmt Anzahl an Werten
    int countValues = Randomizer.GenerateRandomNumber(min, max);
    for (int i = 0; i<countValues; i++)
    {
        int newValue;
        do {
            newValue = Randomizer.GenerateRandomNumber(min, max);
        }
        while (values.Contains(newValue));
        // Füge einzigartige Zahl hinzu
        values.Add(newValue);
    }
    // Erzeuge Wahrscheinlichkeiten
    List<double> valueProbabilities = Randomizer.RandomDistribution(rnd, sum, countValues);
    // Füge Werte in Model ein
    for (int i = 0; i < countValues; i++)
    {
        distribution.Add(values[i], valueProbabilities[i]);
    }

    return distribution;
}
\end{lstlisting}




\subsection{Versand der Nachrichten}
XML Nachrichten werden mit Hilfe der Klassen erzeugt, welche generische Testdaten erzeugen. Diese werden dann mit \emph{Semaine Sender Components} an die Middleware und somit an die Fission verschickt.
An dieser Stelle ist dies Beispielhaft für das PersonModel beschrieben:

\lstset{caption={Semaine Copmponent PersonModelSender},label=PersonModelSender}
\begin{lstlisting}
class PersonModelSender: Component
{
    // Die Sender und Receiver
    private XMLSender _personModelSender;
    private Receiver _requestReceiver;

    // Konstruktor
    public PersonModelSender()
        : base("B3.Budgie.PersonModelSender")
    {
        //Person Model request Receiver
        _requestReceiver = new Receiver("semaine.data.user.request");
        _receivers.AddLast(_requestReceiver);
        // Person Model Sender
        _personModelSender = new XMLSender("semaine.data.user", "XML", this.GetType().Name);
        _senders.AddLast(_personModelSender);
    }

    // Reagiere auf Requests
    protected override void React(SEMAINEMessage message)
    {
        if (message.Text.Equals("UPDATE", StringComparison.OrdinalIgnoreCase)) {
            GenericPersonModel person = new GenericPersonModel("person_id");
            XmlDocument doc = person.Person.XmlRepresentation;
            _personModelSender.SendXML(doc, _meta.CurrentTime);
        }
    } 
}
\end{lstlisting}






\section{Durchführung}


